\newpage
\addcontentsline{toc}{chapter}{Abstract}
\begin{center}
  {\bf Abstract}
\end{center}

There is an abundance of computing power sitting in computer labs waiting
to be harnessed. Previous research in this area has shown promising results
networking clusters of workstations together in order to solve bigger problems
faster at a fraction of the cost for supercomputer time. There are, of course,
challenges to using these sorts of clusters: the communication fabrics linking
these machines aren't necessarily high-performance, and the differences
between individual machines in the cluster require careful load balancing in
order to efficiently use them. These problems have only become greater with
the introduction of acceleration hardware such as GPUs and FPGAs; however, that
hardware also provides even greater computing power at an even lower price
point for those that can work around their idiosyncracies. This dissertation
presents an approach to designing software to effectively utilize these heterogeneous
computing clusters in a modular, extensible manner. We apply it to the development
of a large-scale Neo-Cortical Simulator(NCS) as well as the engineering of a
virtual reality library, caVR.
