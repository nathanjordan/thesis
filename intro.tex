\chapter{Introduction}
\label{chapter:intro}

Experiments in the field of neuroscience have traditionally been conducted using invasive physical methods including \emph{in vivo} (using a living organism) and \emph{in vitro} (in an artificial laboratory environment) methods. These methods, while effective in small-scale experiments, become impractical for performing experiments on a larger scale and are often times quite expensive. To combat these problems, neuroscientists have turned to \emph{in silico} methods to perform experiments, a field known as computational neuroscience. To create neurological experiments in a computer environment, a number of neural simulators that allow researchers to model biologically realistic neurons and neural networks have been created\cite{sejnowski1988computational}. These neural simulators allow researchers to build, run, and output the results of a simulation. While each simulator has the same goal in mind, they each have a different set of features and limitations. Researchers must choose which simulator provides the most appropriate set of functionality for their experiments \cite{brette2007simulation}.

The University of Nevada has created its own neural simulator called NCS, or Neo-Cortical Simulator. NCS tends to focus on larger scale simulations that involve large numbers of neurons and synapses. Performance was also a key consideration when designing NCS, and as a result the simulator was designed to be as fast as possible without adding features that would create additional computational overhead. By ensuring that the simulator is as fast as possible, NCS can run a million-cell model in realtime \cite{hoang2013novel}.

Although very fast and scalable, the NCS simulator by itself does not provide a convenient way for researchers to perform common tasks surrounding the simulation in an intuitive way. NCS provides a thin and feature-limited Python layer that allows it to be used with Python-based programs. This thin layer exposes the minimal features that NCS provides, such as adding neurons and synapses to the simulation. To make the NCS simulator a more useful tool for neuroscientists, additional higher-level features must be added that augment the minimalist approach to designing NCS. These features should provide neuroscientists the ability to easily create models, configure simulations, manage the data generated by simulations, and visualize their results. 

To accomplish this task, we have created a suite of tools that surround the NCS simulator in the form of a web-based interface and higher-level Python interface that gives a much more intuitive medium for interacting with the brain simulator. The web-based interface provides the ability for neuroscientists to create simulations without programming experience, and the Python interface provides high-level access using an easy-to-use Python library. This thesis covers the development and implementation of these tools, as well as the architectural and design choices that were made when creating them. The rest of this thesis is formatted as follows: Chapter 2 discusses background topics surrounding these tools; Chapter 3 details the design and implementation of the software; Chapter 4 presents conclusions and discusses possible future work that could be implemented in the software.